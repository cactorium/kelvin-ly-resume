\documentclass{my_resume}
\usepackage{hyperref}
\usepackage[8pt]{extsizes}
\usepackage[letterpaper, margin=0.125in]{geometry}
\thispagestyle{empty}
\input{contactinfo}

\begin{document}

\contact{Kelvin Ly}{kelvin.ly1618@gmail.com}{\myphonenumber}

\education{University of Central Florida}{2016-Spring 2018}
    {MS, Computer Engineering}{N/A}
\education{University of Central Florida}{2011-2015}
	{BS, Electrical Engineering}{3.905, Magna Cum Laude}

\section{Skills}
\begin{itemize}
    \item Semiprofessional experience with electronics design, PCB layout (KiCad, some Altium Designer), and reverse engineering
    \item Hobbyist level PCB assembly and soldering
    \item Fluent in \textbf{C/C++}, \textbf{Python}, \textbf{Go}, \textbf{Verilog}
    \item Working knowledge of \textbf{x86}/\textbf{x64}/\textbf{MIPS}/\textbf{MSP430} assembly, \textbf{Java}, \textbf{LaTeX}, \textbf{bash}, \textbf{MATLAB}, \textbf{Kicad EDA} Software Suite, \textbf{Multisim}, \textbf{Xilinx ISE}
    \item GitHub user: \url{https://github.com/cactorium}
\end{itemize}

\section{Professional Experience}
\datedsubsection{Fluorometric Instruments Design Engineer, Orlando FL}{September 2017 - present}
\begin{flushleft}
I am working part time to \textbf{design PCBs} for oxygen sensors.
I worked on many of the stages of development for a handful of products, including \textbf{design, testing, parts sourcing, and assembly}.
The work involves designing PCBs to fit mechanical and electrical specifications, as well as \textbf{developing firmware and GUIs} for the devices as necessary.
\end{flushleft}
\datedsubsection{University of Central Florida Undergraduate/Graduate Researcher, Orlando Fl}
    {November 2015 - present}
\begin{flushleft}
The focus on my research here has been on the security of the \textbf{Internet of Things}, more specifically the development of defenses for IoT devices against attacks.
Consequently, much of my work so far has been in \textbf{PCB design and assembly} to develop devices to test out security ideas or provide education on hardware security, and software development to explore ideas in IoT security.
Designs so far have incorporated \textbf{MSP430} and \textbf{Atmel} microcontrollers, along with current work on a primitive \textbf{2.45 GHz radar system} for use in a labmate's project.
I have also previously worked on our submissions for the \textbf{NYU CSAW Embedded Security Competition} '15, '16, and '17 (winning second and first respectively, no win in 2017), which generally involved producing a proof of concept for solving some problem in \textbf{cryptography and security domains}.
These contests led to a wide range of challenges, from writing code to interface with \textbf{MATLAB Simulink} to \textbf{modifying the OpenRISC processor core} and \textbf{patching a GCC backend}.
I am currently being funded by an \textbf{SRC/Intel fellowship}.
\end{flushleft}
\datedsubsection{University of Central Florida Undergraduate Researcher, Orlando FL}
	{December 2014 - March 2015}
\begin{flushleft}
We studied \textbf{feature extraction} from EEG data, focusing on \textbf{SSVEP frequency detection}, using this knowledge in our senior design project, discussed further below.
We used emokit \textbf{Python} library to extract signals from Emotiv EEG headset.
Some work was done with the \textbf{RAVEN II} medical robot running software built on the \textbf{ROS robotics framework}.
\end{flushleft}
\section{Internships}
\datedsubsection{IBM Extreme Blue Intern, RTP NC}{May 2015 - August 2015}
\begin{flushleft}
Here our team worked on \textbf{on-disk encryption} for \textbf{IBM Connections}.
We pioneered work in this direction, producing a proof of concept to pave the way for the actual Connections team to develop.
We used \textbf{JavaScript and Node.js} for the server \textbf{backend}, and modified existing \textbf{Java} and \textbf{Python} code and libraries for various parts of the project.
Our team was organized around modern programming practices, working in an \textbf{agile} team of four, with heavy emphasis on \textbf{test coverage} and \textbf{unit testing}.
\end{flushleft}
\datedsubsection{Google Software Engineer Intern, Chapel Hill NC}{May 2014 - August 2014}
\begin{flushleft}
I worked on the Skia benchmarking team, providing tooling for \textbf{Skia} rendering engine team.
This job involved pipelining the gigabytes of data being produced daily from test bots into a useful visualization for the Skia team.
I learned \textbf{Go}, and contributed code in \textbf{C++}, \textbf{Python}, and \textbf{Go} for both internal and open source projects.
\end{flushleft}
\section{Notable Projects}
\begin{itemize}
    \item \textbf{UCF Lunar Knights} project, electrical/communications/software teams (Martian robotic mining competition)
        \begin{itemize}
          \item Software team lead Fall 2017-Spring 2018, member since 2015
          \item Helped in robot \textbf{assembly}, \textbf{troubleshooting} and \textbf{debugging}
          \item Developed software for \textbf{robot simulation and testing}, mainly through providing wrappers
            in \textbf{ROS} for \textbf{gazebo}
          \item Designed \textbf{CAN interfacing board} with \textbf{high density connectors} to mate with Nvidia's Jetson TX2
          \item Led efforts in \textbf{robotic autonomous navigation}
        \end{itemize}
      \item Senior design project (mind-controlled wheelchair)
        \begin{itemize}
            \item Led high-level hardware system design
            \item Designed and layed out circuits for all high-level modules using \textbf{KiCAD} EDA software
            \item Research into \textbf{signal processing} for \textbf{feature extraction} with respect to applications in \textbf{brain-computer interfaces}
            % \item Created and designed laser cut design to create gimbal to control wheelchair joystick
            % \item Wrote \textbf{assembly} for the \textbf{MSP430} to test the gimbal
        \end{itemize}
\end{itemize}
\end{document}
